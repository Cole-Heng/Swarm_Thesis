\chapter{Introduction}
\label{sec:intro}

Across many disciplines, in businesses, government projects, and academia, robots are being used to improve efficiency of certain tasks. Robots have been invited into our homes to clean the floors, entertain children in the form of toys, and mow our lawns.  In manufacturing and goods movement, robots are performing many tasks to improve these needed tasks.  Robotics is a broad field that includes countless form factors and abilities. One growing field of research and development within robotics is multi-agent systems and swarm robotics.

Swarm robotics is the control of multiple robots, also known as agents, in a manner that simulates the logic and movement of a swarm of biological animals or insects. These agents have individual control of themselves, but collaborate with their neighbors to accomplish a common goal. In swarm applications, the robots are able to move autonomously and are simple in design, often with only a few necessary sensors and actuators. The simplicity of their design allows swarms of robots to be more easily scaled in size and maintained.

Swarm robotics combines multi-agent systems with swarm intelligence. Swarm intelligence refers to emergent behavior similar to biological swarms, such as insects, fish, and birds, that gather information from their own senses and the interactions they have with their neighbors, but do not possess any omnipotent knowledge of the world around them. This is applied to algorithms that mimic these biological features onto a group of agents working within the same system.

Many applications for swarms of robots are emerging and have been proposed \cite{spezzano2019special}. Common uses for swarms of robots are to perform tasks in places where humans cannot or should not go. Examples of these applications are detecting chemical leaks, search and rescue in dangerous terrain, and working in radioactive environments. Industrial robots are also being used to replace humans in very repetitive, simple, or labor-intensive tasks, and swarms of robots are no different. Swarms are being used in warehouses to move products \cite{poudel2013coordinating} and on farms \cite{spezzano2019special} to increase the amount of simultaneous work that is able to be done.

Across all real-world robotics systems, safety is a central concern during the design and implementation of a system. ``Safety" can have different meanings depending on context. A robot may need to be gentle and precise when manipulating manufactured goods, avoid destruction to the physical environment and infrastructure in which it works, avoid risk to human life and injury, and prevent damage to itself and other robots it interacts with. Safety-critical systems are systems in which faults and failures can result in significant damage to any of the aforementioned areas \cite{knight2002safety}, with systems posing a risk to human safety being the most critical. There are many ways to make systems safe through hardware, software, and infrastructure, with more critical systems implementing more layers of safety. 

\section{Proposed Research}

Our research seeks to implement a swarm algorithm, in simulation and on a real-world system, and analyze the efficacy of the algorithm to keep the swarm safe. We extend a swarming algorithm with control barrier functions to make comparisons between the two algorithms in an attempt to make a simple algorithm safety-critical. Many algorithms have been developed to implement a swarm in simulation, real-world, or both. One simple algorithm is the Boids Algorithm developed by Craig Reynolds in 1987 \cite{reynolds1987flocks}, also called the Reynolds Boids or Reynolds Algorithm. This algorithm simulates the flocking behavior of birds by combining three base actions described by Reynolds: Separation, Alignment, and Cohesion. The effects of each parameter and a more in-depth explanation of the Boids Algorithm can be found in Chapter \ref{ch:back}, Section \ref{sec:boids}. The Boids Algorithm has been applied to robot control in many scenarios \cite{kasmarik2020autonomous,clark2012flight,jakimovski2008swarm}, both in simulation and on physical systems, and this thesis will do the same. 

This thesis aims to create a safety-critical swarm system using a simple base algorithm and extending it. The Boids Algorithm is the base algorithm for the control of our swarm, chosen for its relatively simple control paradigm. Our research aims to investigate the safety of this simple control algorithm as compared to an extended version of Boids (which we call Extended\_Boids) and a version of Boids modified to include control barrier functions. Speed, both in terms of time for the swarm to complete its task and the computational time to simulate, as well as the number of agents that reach their goal without a collision, is studied for comparing our algorithms. The safety goal is to achieve all agents at their destination without any collisions between two agents, an agent and the boundaries, or an agent and an obstacle. 

There are three systems we will investigate and compare. First, is the Boids Algorithm as described by Reynolds \cite{reynolds1987flocks} that just has agents flock without purpose. Second, the Extended\_Boids Algorithm that uses the same three rules, but with added functionality by adding what we call ``ghost boids" or ATONs (Aids TO Navigation). Ghost boids do not move but have influence on the movement of other boids, allowing an invisible influence that can be used to add obstacle avoidance and goal-seeking by positioning the ghost boids in certain locations, all without modifying the control algorithm. Lastly, we implement control barrier functions on top of the Extended\_Boids Algorithm which then allows the boids to act normally unless they are approaching unsafe conditions, at which point the control will begin to focus more on safety. We will investigate the performance of all three systems and compare them. After doing this in simulation, we aim to implement these algorithms on a physical swarm to investigate the success of the algorithms in a real-world environment.

\section{Contributions Of Our Work}

Our work has three main contributions:
\begin{enumerate}
    \item The novel use of ATONs and ghost boids to extend the Boids Algorithm to include goal-seeking and obstacle avoidance without modifying the base algorithm.
    \item The creation of a safety-critical swarm system that can be implemented in simulation or a real system through the use of control barrier functions.
    \item A methodology to analyze and compare of the performance of these algorithms.
\end{enumerate}

\section{Organization}

%Swarm robotics is an ever-growing section in the field of robotics. With applications in business, academia, and government environments, swarms of robots have individual control and autonomy, but collaborate for a common goal. Mimicking the behavior of biological swarms, robot swarms are usually simple in design to be low in cost and scalable. Our research investigates the safety-critical control of a robot swarm by simulating algorithms built on top of a simple control algorithm, the Boids Algorithm. Lastly, the algorithms will be implemented on a physical swarm of robots to investigate their real-world successes and failures.
The remainder of this thesis is organized as follows. In Chapter \ref{ch:back}, we provide the relevant background and state of the art. In Chapter~\ref{ch:3} Section~\ref{sec:work}  we discuss the progress we have made in this work. Finally, in Section \ref{sec:future} we provide a description of the work we will do to complete this thesis and a timeline for the proposed work.